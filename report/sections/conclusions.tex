\chapter{Conclusions}
This project allowed us to put Domain Driven Design methodologies into practice to develop a state-of-the-art system that brings potential benefit to the company. The ambition of the project was also to understand how such methodologies can lend themselves to modeling digital twins to support domain-defined processes.

Although DDD methodologies have proven to be useful and strategic for modeling the domain, these features have come somewhat short in modeling digital twins. We soon realized that in this particular domain digital twins are particularly coupled with devices and machinery for cheese production and packaging, domain experts did not always know the details of how the machinery works, frustrating some DDD-defined methodologies, such as event storming. Moreover, domain experts do not perceive Digital Twins since their conception is to support processes by improving them. However, knowledge crunching sessions allowed us to understand the domain in depth and identify weaknesses in current processes rather well and identify where and how digital twins could be a real advantage.

DDD methodologies, on the other hand, proved to be critical in managing the relationships between digital twins and the various bounded contexts that emerged in the early design phases. We proceeded by modeling the digital twins as if they were external bounded contexts and then defining the relationships that exist between them. This has, from our point of view, simplified the way bounded contexts and digital twins communicate. Furthermore, by exploiting the principle of DTOs we integrated the information in the digital twins with the concepts proper to each bounded context avoiding inconsistent states or domain corruption.

This project allowed us to delve into technologies such as Eclipse Ditto, examining its architecture the main functionalities as well as verifying its operation in a real scenario such as the dairy.
We familiarized ourselves with the WoT specification for modeling objects, in particular, we took advantage of the Thing Model specification that integrates seamlessly with Eclipse Ditto by automatically generating a Thing Description for each instance of a digital twin, eliminating any possible error derived from duplication of models.
The models created were annotated with specific ontologies to enable a semantic layer, useful for making logical inferences and deriving further information.

Despite the limited time, the results obtained were more than satisfactory. At the end of the project, the prototype was shown to domain experts who were positively surprised and glimpsed the potential of the system thus modeled. In conclusion, we are satisfied with both the results obtained and the feedback received, as well as having learned new insights about the world of digital twins and DDD.

\section{Future works}
Several interventions are needed to get the system up and running. Much logic that is needed for communication with the outside world has been mocked up for reasons of time.

The modeling of the digital twins covers much of the domain concepts that have emerged and the related needs; however, we expect that refactoring is needed to better fit real-world scenarios.

Nevertheless, the project is based on a solid foundation, so all the remaining work is mostly implementation.

\subsubsection{Simulation of the digital twins}
One of the reasons digital twins came into being is to exploit them to carry out simulations and thus generate useful data for the real world.

One scenario we envisioned where digital twins can be used in simulation is the context of cheese packaging.

As described in the section on the digital twin of the packaging machine, the digital twin is also exploited, among many other things, to make predictions about machine failures in order to take timely action and minimize downtime. In this case, the simulation could focus on simulating product packaging by exploiting historical data and based on the results determine when the machinery is about to fail.
