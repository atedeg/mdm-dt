\chapter{Bounded Contexts}


\section{Milk Planning}
% TODO: sistemare prevedendo i DT 
Every Saturday Raffaella has to estimate the quintals of milk necessary to produce all products
for the following week.
She makes this estimate by taking into account the following factors:

\begin{itemize}
    \item the quintals of milk processed in the same period of the previous year
    \item the quintals of milk needed by the products that have to be produced in the following week
    (this is made by reading from a recipe book the yield of milk to produce a quintal of a given product)
    \item the current stock
    \item the quintals of milk already in stock

\end{itemize}

After the estimate is complete the restocking B.C. is notified of the result so
that it can make a milk order

\begin{table}[H]
    \centering
    \begin{tabular}{|p{0.2\textwidth}|*{3}{>{\arraybackslash}p{0.7\textwidth}|}}
        \hline
        Term              & Definition                                                                                                                                                                                 \\ \hline
        Processed milk    & The quintals of milk are processed in order to produce cheese.                                                                                                                             \\ \hline
        Quintals of milk  & A quantity of milk expressed in quintals.                                                                                                                                                  \\ \hline
        Yield             & A decimal that represents the yield of milk when producing a given cheese type: i.e. to produce n quintals of a given cheese type, yield of cheese type * n quintals of milk must be used. \\ \hline
        Recipe book       & It defines, for each cheese type, the yield of milk when producing it.                                                                                                                     \\ \hline
        Stock             & It defines, for each product, the quantity available in stock.                                                                                                                             \\ \hline
        Stocked quantity  & A quantity of a stocked product, it may also be zero.                                                                                                                                      \\ \hline
        Quantity          & A quantity of something.                                                                                                                                                                   \\ \hline
        Requested product & A product requested in a given quantity that has to be produced by the given date.                                                                                                         \\ \hline
    \end{tabular}
    \caption{Ubiquitous Language}
\end{table}

\begin{table}[H]
    \centering
    \begin{tabular}{|p{0.2\textwidth}|*{3}{>{\arraybackslash}p{0.7\textwidth}|}}
        \hline
        Event          & Description                                                                             \\ \hline
        Received order & Event representing an order placed used to estimate the quintals of milk to be ordered. \\ 
        \hline
        % TODO: aggiungere eventi che arrivano dai DT
    \end{tabular}
    \caption{Incoming events}
\end{table}

\begin{table}[H]
    \centering
    \begin{tabular}{|p{0.2\textwidth}|*{3}{>{\arraybackslash}p{0.7\textwidth}|}}
    \hline
        Event      & Description                                                                                                 \\ \hline
        Order milk & Event to order the quintals of milk needed for the next week. This event is emitted every week on saturday. \\ \hline
    \end{tabular}
    \caption{Outgoing events}
\end{table}

\section{Milk Tank DT}
A milk tank is a machine that stores milk.
It has a sensor that measures the quantity of milk in the tank as well as temperature and pH value sensors.
It also has a valve that allows the milk to be pumped out of the tank.
The tanks send the milk level to Milk Planning B.C. in order to estimate the quintals of milk necessary to produce
all products for the following week.
Temperature and pH values are used to analyze the milk quality.
Every time one of the two values above becomes out of range, a warning is sent to Alerts B.C.

\begin{table}[H]
    \centering
    \begin{tabular}{|p{0.2\textwidth}|*{3}{>{\arraybackslash}p{0.7\textwidth}|}}
        \hline
        Term              & Definition      \\ \hline
        Quintals of milk  & A quantity of milk expressed in quintals.  \\ \hline
        Temperature             & The current temperature of the milk in the tank. \\ \hline
        PH       & The current pH value of the milk in the tank.      \\ \hline
    \end{tabular}
    \caption{Ubiquitous Language}
\end{table}

\begin{table}[H]
    \centering
    \begin{tabular}{|p{0.2\textwidth}|*{3}{>{\arraybackslash}p{0.7\textwidth}|}}
        \hline
        Term               & Definition                                                               \\ \hline
        Open the Valve    & Fired when the tank's valve needs to be opened.     \\ \hline
        PH out of range & An event emitted when the pH value is out of the required range. \\ \hline
        Temperature out of range & An event emitted when the Temperature value is out of the required range. \\ \hline
    \end{tabular}
    \caption{Outgoing events}
\end{table}

\section{Stocking}
% TODO: sistemare prevedendo i DT
After a batch of cheeses has ripened for the required amount of time,
one of them is selected to perform quality assurance.
Its result could be either positive or negative.
The former results in the cheeses being wrapped, labeled and put in the refrigeration room;
the latter results in the entire batch being discarded.

To label a cheese the worker has to weigh it, and an automated system will print an appropriate label.

\begin{table}[H]
    \centering
    \begin{tabular}{|p{0.2\textwidth}|*{3}{>{\arraybackslash}p{0.7\textwidth}|}}
        \hline
        Term                  & Definition                                                                                                                                       \\ \hline
        Available stock       & The currently available products in stock; each one is available in a certain quantity (that could also be zero if the product is out-of-stock). \\ \hline
        Desired stock         & The desired quantity of each product that should always be in stock in order to have a safe margin to keep order fulfillment going.              \\ \hline
        Available quantity    & The quantity in stock of a certain product.                                                                                                      \\ \hline
        Desired quantity      & The desired quantity of a certain product to be in stock.                                                                                        \\ \hline
        Missing quantity      & The required quantity of a certain product to reach the desired stock level.                                                                     \\ \hline
        Batch                 & A batch of products of a certain type, uniquely identified by an ID, which hasn't been quality assured.                                          \\ \hline
        Quality assured batch & A batch of products of a certain type uniquely identified by an ID, which has undergone quality assurance.                                       \\ \hline
        Labelled product      & A product with its respective quantity and the ID of the batch it belongs to.                                                                    \\ \hline
    \end{tabular}
    \caption{Ubiquitous Language}
\end{table}

\begin{table}[H]
    \centering
    \begin{tabular}{|p{0.2\textwidth}|*{3}{>{\arraybackslash}p{0.7\textwidth}|}}
        \hline
        Term                              & Definition                                            \\ \hline
        Batch ready for quality assurance & Received when a batch is ready for quality assurance. \\ \hline
        Product removed from stock        & Received when a product is removed from the stock.    \\ \hline
        New batch                         & Received when a batch is created.                     \\ \hline
    % TODO: aggiungere eventi che arrivano dai DT
    \end{tabular}
    \caption{Incoming events}
\end{table}

\begin{table}[H]
    \centering
    \begin{tabular}{|p{0.2\textwidth}|*{3}{>{\arraybackslash}p{0.7\textwidth}|}}
        \hline
        Term               & Definition                                                               \\ \hline
        Product stocked    & Fired when a label is printed for a product, which is then stocked.      \\ \hline
        Product palletized & An event emitted when a product is successfully palletized for an order. \\ \hline
    \end{tabular}
    \caption{Outgoing events}
\end{table}


\section{Scale DT}
A scale is a machine that provides the weight of the products.
It is used to weigh the products before they are stocked as a part of the quality assurance process.
Provides information about the current batch it is processing, the mean weight of the products in the batch and the standard deviation of the weight of the products in the batch.
It iforms that all of the packages of a batch to the Stocking B.C. the number of dropped packages (that is increased either if a product is overweight or underweight) with information about the corresponding batch, and the batch completed event with the corresponding batch.

\begin{table}[H]
    \centering
    \begin{tabular}{|p{0.2\textwidth}|*{3}{>{\arraybackslash}p{0.7\textwidth}|}}
        \hline
        Term              & Definition      \\ \hline
        Current Batch  & The current batch is processing. \\ \hline
        Mean weight             & The mean weight of the products in a batch. \\ \hline
        Standard deviation weight       & The standard deviation weight of the products in a batch.      \\ \hline
    \end{tabular}
    \caption{Ubiquitous Language}
\end{table}

\begin{table}[H]
    \centering
    \begin{tabular}{|p{0.2\textwidth}|*{3}{>{\arraybackslash}p{0.7\textwidth}|}}
        \hline
        Term               & Definition                                                               \\ \hline
        Batch completed & An event emitted when the packages of a specific batch are all scaled. It contains the number of dropped packages. \\ \hline
    \end{tabular}
    \caption{Outgoing events}
\end{table}


\section{Packaging Machine DT}
% storiella di cosa è la packaging machine


\section{Metal Detector DT}
Il fatto che il metal rilevi componenti metalliche è molto raro.
% questa cosa di seguito la mettiamo tra le motivazioni:
% Se accade  che il metal rilevi qualcosa voglio anche controllare i macchinari per vedere se c'è sono deterioramento etc.
Per tanto se l'evento accade , il batch viene scartato per ulteriori accertamenti sulla qualità dei prodotti.
Il batch scartato non viene propagato allo step successivo(bilancia).
L'evento di scarto del batch viene propagato allo Stocking.

\section{Alerts}
% storiella di cosa sono gli alerts

\section{More bounded contexts}
In this report we omitted the other bounded contexts that were identified for the case study since they are not part of the scope of this project.

However, their full description can be found here: \url{https://atedeg.dev/mdm/_docs/index.html}.
