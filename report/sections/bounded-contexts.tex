\chapter{Bounded Contexts}
\section{Milk Planning}
% TODO: sistemare prevedendo i DT ? 
Every Saturday Raffaella has to estimate the quintals of milk necessary to produce all products
for the following week.
She makes this estimate by taking into account the following factors:


\begin{itemize}
    \item the quintals of milk processed in the same period of the previous year
    \item the quintals of milk needed by the products that have to be produced in the following week
    (this is made by reading from a recipe book the yield of milk to produce a quintal of a given product)
    \item the current stock
    \item the quintals of milk already in stock
     
\end{itemize}
  
After the estimate is complete the restocking B.C. is notified of the result so
that it can make a milk order

\begin{table}[H]
    \centering
    \begin{tabular}{p{0.2\textwidth}*{3}{>{\arraybackslash}p{0.8\textwidth}}}
    \hline
        Term & Definition \\ \hline
        Processed milk & The quintals of milk are processed in order to produce cheese. \\ \hline
        Quintals of milk & A quantity of milk expressed in quintals. \\ \hline
        Yield & A decimal that represents the yield of milk when producing a given cheese type: i.e. to produce n quintals of a given cheese type, yield of cheese type * n quintals of milk must be used. \\ \hline
        Recipe book & It defines, for each cheese type, the yield of milk when producing it. \\ \hline
        Stock & It defines, for each product, the quantity available in stock. \\ \hline
        Stocked quantity & A quantity of a stocked product, it may also be zero. \\ \hline
        Quantity & A quantity of something. \\ \hline
        Requested product & A product requested in a given quantity that has to be produced by the given date. \\ \hline
    \end{tabular}
    \caption{Ubiquitous Language}
\end{table}

\begin{table}[H]
    \centering
    \begin{tabular}{p{0.2\textwidth}*{3}{>{\arraybackslash}p{0.8\textwidth}}}
    \hline
        Event & Description \\ \hline
        Received order & Event representing an order placed used to estimate the quintals of milk to be ordered. \\ \hline
    \end{tabular}
    \caption{Incoming events}
\end{table}

\begin{table}[H]
    \centering
    \begin{tabular}{p{0.2\textwidth}*{3}{>{\arraybackslash}p{0.8\textwidth}}}
    \hline
        Event & Description \\ \hline
        Order milk & Event to order the quintals of milk needed for the next week. This event is emitted every week on saturday. \\ \hline
    \end{tabular}
    \caption{Outgoing events}
\end{table}

\section{Milk Tank}
% storiella di cosa è il milk tank

\section{Stocking}
% TODO: sistemare prevedendo i DT ? 
After a batch of cheeses has ripened for the required amount of time,
one of them is selected to perform quality assurance.
Its result could be either positive or negative.
The former results in the cheeses being wrapped, labeled and put in the refrigeration room;
the latter results in the entire batch being discarded.

To label a cheese the worker has to weigh it, and an automated system will print an appropriate label.

\begin{table}[H]
    \centering
    \begin{tabular}{p{0.2\textwidth}*{3}{>{\arraybackslash}p{0.8\textwidth}}}
    \hline
        Term & Definition \\ \hline
        Available stock & The currently available products in stock; each one is available in a certain quantity (that could also be zero if the product is out-of-stock). \\ \hline
        Desired stock & The desired quantity of each product that should always be in stock in order to have a safe margin to keep order fulfillment going. \\ \hline
        Available quantity & The quantity in stock of a certain product. \\ \hline
        Desired quantity & The desired quantity of a certain product to be in stock. \\ \hline
        Missing quantity & The required quantity of a certain product to reach the desired stock level. \\ \hline
        Batch & A batch of products of a certain type, uniquely identified by an ID, which hasn't been quality assured. \\ \hline
        Quality assured batch & A batch of products of a certain type uniquely identified by an ID, which has undergone quality assurance. \\ \hline
        Labelled product & A product with its respective quantity and the ID of the batch it belongs to. \\ \hline
    \end{tabular}
    \caption{Ubiquitous Language}
\end{table}

\begin{table}[H]
    \centering
    \begin{tabular}{p{0.2\textwidth}*{3}{>{\arraybackslash}p{0.8\textwidth}}}
    \hline
        Term & Definition \\ \hline
        Batch ready for quality assurance & Received when a batch is ready for quality assurance. \\ \hline
        Product removed from stock & Received when a product is removed from the stock. \\ \hline
        New batch & Received when a batch is created. \\ \hline
    \end{tabular}
    \caption{Incoming events}
\end{table}

\begin{table}[H]
    \centering
    \begin{tabular}{p{0.2\textwidth}*{3}{>{\arraybackslash}p{0.8\textwidth}}}
    \hline
        Term & Definition \\ \hline
        Product stocked & Fired when a label is printed for a product, which is then stocked. \\ \hline
        Product palletized & An event emitted when a product is successfully palletized for an order. \\ \hline
    \end{tabular}
    \caption{Outgoing events}
\end{table}

\section{Scale}
% storiella di cosa è la scale

\section{Packaging Machine}
% storiella di cosa è la packaging machine

\section{Metal Detector}
% storiella di cosa è il metal detector


\section{More bounded contexts}
In this report we omitted the other bounded contexts that were identified for the case study since they are not part of the scope of this project.

However, their full description can be found here: \url{https://atedeg.dev/mdm/_docs/index.html}.
