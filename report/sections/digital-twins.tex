\chapter{Digital Twins}
In industrial contexts, digital twins can be useful for several purposes.
In our case study, we identified the digital twins as the ``Process Manufacturing'' use case.

In fact, the area of application for this kind of digital twins are:
\begin{itemize}
	\item improve product quality and process optimization
	\item predict equipment failures with machine learning models based on real-time data and models and on historical failure data
	\item monitor real-time compliance with safety and regulatory requirements
\end{itemize}
Below we will analyze, for each digital twin, the context in which it lives and which of the above characteristics it falls into to improve business processes.

Digital twins are commonly divided into subtypes that sometimes include: digital twin prototype (DTP), digital twin instance (DTI), and digital twin aggregate (DTA).
The DTP consists of the designs, analyses, and processes that realize a physical product.
The DTP exists before there is a physical product so is not a digital twin by definition.
The DTI is the digital twin of each instance of the product once it is manufactured.
The DTA is the aggregation of DTIs whose data and information can be used for interrogation about the physical product, prognostics and learning.
The specific information contained in the digital twins is driven by use cases.

In this project, we dealt mainly with DTI because of the industrial and manufacturing nature of the case study.
However, we believe that the use and modeling of DTA-type digital twins could be useful to improve the quality and production efficiency of products.

DTA-type digital twins were not examined in this project because of the current limitations of digital twin-enabling platforms.
We are not aware of any platforms that natively model the concept of digital twin aggregation, so for the sake of simplicity, we decided not to implement such a concept.


Below we illustrate the digital twin modeling that is part of the project; in particular, we will examine the functionalities they cover and what data they
manipulate as well as make considerations about the strategic impact they have on business production processes.

\section{Digital twins of machines}
Following the domain analysis with experts, we focused on what were the most critical parts of the whole process of making cheese.
This analysis showed that the steps that affect productivity the most are product packaging and the related quality assurance process.

Domain experts also explained to us how milk quality is critical, and currently, no system monitors its status in real-time.
Ad-hoc analyses are done before each production to verify the goodness of the milk to be used in production.
In this regard, it would be good to have a system in place that monitors milk values in real-time so that action can be taken if the need arises,
thus avoiding analysis before using it for production.


As a result of the considerations and weaknesses raised by domain experts, we identified the following digital twins:
\textit{milk-tank} that monitors the parameters of the stored milk; \textit{packing-machine} that monitors the parameters that are then used to carry
out predictive maintenance; and finally, \textit{metal-detector} and \textit{scales} to improve the quality assurance process by detecting contamination in products and defects in product weight, respectively.

\subsection{Milk Tank}
As mentioned earlier, the digital twin we individuated for the Milk Planning subdomain is the Milk Tank Digital Twin.
The need for the adoption of this digital twin is to have real-time monitoring of the milk quantity and quality in the tanks.
The digital twin will be able to provide the Milk Planning subdomain with information about the milk quantity in the tanks, which will be used to estimate the amount of milk to order each week.

Moreover, the tank will provide information about the pH and the temperature of the milk, which will be used to decide if the milk is suitable for the production of a specific cheese.


\subsubsection{Thing Model}

\lstinputlisting[label={lst:milk-tank-tm},caption={Milk tank Thing Model.}]{lst/milk-tank.tm.jsonld}

As you can see from the listing~\ref{lst:milk-tank-tm}, two static properties were defined: \textit{serialNo} and \textit{Manufacturer}.
Those two properties represent the serial number of the tank and the manufacturer, respectively.
Since those properties come with the tank, they are read-only.

Another read-only property is \textit{availableMilk}, which represents the level of milk in the tank. Since the level is calculated directly by the
tank, it is read-only so that no external operation could alter the value. This property is used directly by the \textit{milk-planning} to
plan the reordering of milk.

The only available action is \textit{openValve}, which is used to open the valve and start the production process. The action is represented by a boolean value representing the current state of the valve.

This particular digital twin is characterized by the fact that is linked by other two devices: \textit{temperature-sensor} and \textit{ph-sensor}.
Those two sensors are independent devices that are attached to the tank and are used to monitor the temperature and the pH of the milk.
In the \textbf{Thing Model} specification the connection is represented by the \texttt{links} construct showing that the ``root thing'' is connected
with other ``things''.
This kind of connection enables semantic information about how things are connected and which kinds of relationships occur between them.

The \textit{temperature} sensor defines the following properties: \textit{temperature}, \textit{min-temp} and \textit{max-temp}. The former represents the current temperature from the sensor; the last two properties are useful to define the range of temperature that is considered normal.
In fact, if the \textit{temperature} value is outside the range defined by \textit{min-temp} and \textit{max-temp}, an alarm is raised through the event \textit{temperature-out-of-range}. The event is annotated with an ontology term that represents the concept of temperature so that a reasoner
could infer the meaning of the event.

The temperature sensor is represented as a separate thing model. This enables us to reuse the same definition for other devices, for example for the
cutter of the \textit{packaging-machine}. In this way, we can reuse the same definition for different devices and avoid duplication of models.

The \textit{ph} sensor is quite similar to the temperature sensor, but it defines the following properties: \textit{ph}, \textit{min-ph} and \textit{max-ph}. The former represents the current pH value from the sensor; the latter two properties are useful to define the range of pH that is considered normal. Also the \textit{ph} sensor defines an event \textit{ph-out-of-range} that is raised when the \textit{ph} value is outside the range defined by \textit{min-ph} and \textit{max-ph}.
The event is annotated with an ontology term that represents the concept of pH so that a reasoner could infer the meaning of the event. 

\subsection{Packaging machine}
The packaging machine is a very important machine in the packaging process as well as very complex and expensive.
It is also fairly prone to breakdowns and this can cause a lot of problems in the packaging process.
For these reasons, the introduction of a digital twin that can monitor the machine and predict when it will break down can be considered crucial.


\subsubsection{Thing Model}

\begin{lstlisting}[caption={Properties of the packaging machine Thing Model.},label={lst:prop-pack}]
"properties": {
	"manufacturer": {
		"title": "Manufacturer",
		"observable": false,
		"readOnly": true,
		"type": "string",
		"forms": []
	},
	"serialNo": {
		"title": "Serial number",
		"type": "string",
		"readOnly": true
	},
	"currentBatch": {
		"title": "Current Batch",
		"type": "string",
	}
}
\end{lstlisting}

In the listing~\ref{lst:prop-pack}, we can see the properties of the digital twin of the packaging machine. In particular, we define the
\textit{manufacturer} and the \textit{serialNo} of the machine, which are read-only properties;
the \textit{currentBatch} represents the current batch that is being packaged, this property will be set by the service used for managing the stock.
The \textit{currentBatch} property is used by the machine to configure its parameters such as cutting temperature, speed, type of packaging, length
of the packaging and so on.

The listing~\ref{lst:action-pack} shows the actions of the digital twin of the packaging machine. In particular, we define the \textit{start} action
which will be used to start the packing process following the parameter set by the \textit{currentBatch} property.
This action could be triggered by a service that automatically starts the packaging process or reflects the manual intervention of an operator that starts the process manually.
The \textit{start-packaging} is represented by a boolean value that represents the current state of the machine.

\begin{lstlisting}[caption={Actions of the packaging machine Thing Model.},label={lst:action-pack}]
"actions": {
	"start-packaging": {
		"title": "Start Packaging",
		"description": "Start (or stop) the packaging machine based on the boolean value",
		"type": "boolean"
	}
}
\end{lstlisting}

In the listing~\ref{lst:event-pack}, we can see the events of the digital twin of the packaging machine.

\begin{lstlisting}[caption={Events of the packaging machine Thing Model.},label={lst:event-pack}]
"events": {
	"packaging-started": {
		"title": "Packaging Started",
		"data": {
			"@type": "Boolean",
			"title": "Packaging Started",
			"type": "boolean"
		}
	},
	"packagingMachine-failure": {
		"title": "Packaging machine failure",
		"type": "object",
		"properties": {
			"batch-id": {
				"type": "string"
			},
			"cutter-temperature": {
				"type": "integer"
			}
		}
	},
	"package-damaged": {
		"title": "Package damaged",
		"type": "object",
		"properties": {
			"batch-id": {
				"type": "string"
			},
			"cutter-temperature": {
				"type": "integer"
			}
		}
	}
}
\end{lstlisting}

The events that the packaging machine could emit are distinct in two categories: the first category is related to the state of the machine (\textit{packaging-started}), and the second category is related to the failure of the machine (\textit{packagingMachine-failure} and \textit{package-damaged}). The first event is raised when the machine starts the packaging process, and the second event is raised when the machine fails or the packaging is damaged.

The \textit{packaging-started} event will be used by the service that manages the stock to update the status of the batch that is being packaged.
This information is useful to keep track of the batches that are being packaged and to know when the packaging process is completed. In this way it
is possible to create accurate reports about the packaging process.

The \textit{packagingMachine-failure} event is raised when the machine fails. At the moment this kind of failure is tight only to the cutter temperature because is the main component prone to breakage. This event could (and should) in a realistic scenario be enriched with more information about the failure, such as the exact component that failed, the time of the failure, and so on. In this way, it is possible to create a more accurate digital twin of the machine and to predict the failure of the machine.

The \textit{package-damaged} event is raised when the packaging is damaged. This event is raised when the machine fails to package correctly a cheese in the batch. This event is captured by the service that manages all the reports to create a report about the damaged packages.
Moreover, the enrichment of this event with additional information about the failure could be useful for the predictive maintenance of the machine
because is likely that the machine suffers from excessive mechanical stress that reduces the RUL of the machine.

% cutterTemperature viene passato all'algoritmo che in caso di packagingMachineFailure o packageDamaged poi vede che magari la temperatura era troppo alta e capisce qual era il problema

% riceve le info su quale lotto deve iniziare a confezionare così da impostare automaticamente i parametri della macchina.
% tramite algoritmo di machine learning, la macchina predice il guasto e lo comunica tramite alert.
% I parametri che valuta sono: numero di chiusure al minuto per tipo di prodotto, numero di confezioni danneggiate, temperatura resistenza saldatore (cutter).

\subsection{Metal detector}
The metal detector is a device that is used to detect metal objects in packages. It is used to ensure that the packages are free of metal objects, which could be dangerous for the consumer.
The digital twin of the metal detector will be able to detect if there are any anomalies in the packages.

\subsubsection{Thing Model}

\begin{lstlisting}[caption={Properties of the metal detector Thing Model.},label={lst:prop-metal}]
"properties": {
	"manufacturer": {
		"title": "Manufacturer",
		"type": "string",
		"readOnly": true
	},
	"serialNo": {
		"title": "Serial number",
		"type": "string",
		"readOnly": true
	},
	"currentBatch": {
		"title": "The current batch",
		"type": "string"
	}
}
\end{lstlisting}

The digital twin of the metal detector has the same properties as the digital twin of the packaging machine, in particular, it has the \textit{manufacturer} and the \textit{serialNo} of the machine, which are read-only properties; the \textit{currentBatch} property is used to configure the machine so that it knows which batch it has to check. The described properties are represented in the listing~\ref{lst:prop-metal} via the properties
scope in the thing model specification.

\begin{lstlisting}[caption={Events of the metal detector Thing Model.},label={lst:event-metal}]
"events": {
	"dropped-packages": {
		"title": "The dropped packages",
		"description": "Is raised when a batch is completed but some packages fail the check.",
		"type": "object",
		"properties": {
			"batch-id": { "type": "string" },
			"dropped-packages": { "type": "integer" }
		}
	}
}
\end{lstlisting}

As can be observed in the listing~\ref{lst:event-metal}, the metal detector has only one event, which is raised when a batch is completed but some packages fail the check. This event is captured by the service that manages all the reports to create a report about the dropped packages.
The event carries information about the batch that has been checked and the number of packages that have been dropped.
This kind of event is captured by the stocking service that determines if the batch is still usable or if it has to be discarded.

This digital twin has no actions because it is not possible to interact with the machine: the machine works in a ``continual flow'' fashion and no human intervention is required. The only configuration that can be done is to set the batch that has to be checked.

\subsection{Scale}
We decided to implement a digital twin for the Scale that detects underweight and overweight packages.
It supports the Stocking B.C. in the quality assurance process.

\subsubsection{Thing Model}

\begin{lstlisting}[caption={Properties of the scale Thing Model.},label={lst:prop-scale}]
"properties": {
	"manufacturer": {
		"title": "Manufacturer",
		"type": "string",
		"readOnly": true
	},
	"serialNo": {
		"title": "Serial number",
		"type": "string",
		"readOnly": true
	},
	"currentBatch": {
		"title": "Current Batch",
		"type": "string",
		"readOnly": true
	},
	"meanWeight": {
		"title": "Mean weight",
		"type": "number",
		"readOnly": true
	},
	"stdDevWeight": {
		"title": "Standard deviation weight",
		"type": "number",
		"readOnly": true
	}
}
\end{lstlisting}

The digital twin of the scale has the same properties as the digital twin of the packaging machine, in particular, it has the \textit{manufacturer} and the \textit{serialNo} of the machine, which are read-only properties; the \textit{currentBatch} property is used to configure the machine so that it knows which batch it has to check. The \textit{meanWeight} and \textit{stdDevWeight} properties are used to store the mean and standard deviation of the weight of the packages in the batch. The described properties are represented in the listing~\ref{lst:prop-scale}.

The scale during its operational cycle produces a report that contains the mean and standard deviation of the weight of the packages in the batch. This report is captured by the service that manages all the reports to create a report about the weight of the packages in the batch.

\begin{lstlisting}[caption={Events of the scale Thing Model.},label={lst:event-scale}]
"events": {
	"dropped-packages": {
		"title": "Report the number of dropped packages",
		"description": "Report the number of too much eavy/light packages that are dropped.",
		"type": "object",
		"properties": {
			"batch-id": { "type": "string" },
			"dropped-packages": { "type": "integer" }
		}
	},
	"batch-completed": {
		"title": "Batch is completed",
		"description": "Report that the scaling of a batch has been completed",
		"type": "object",
		"properties": {
			"completed": { "type": "boolean" },
			"batch-id": { "type": "string" },
			"dropped-packages": { "type": "integer" }
		}
	}
}
\end{lstlisting}

In the listing~\ref{lst:event-scale} are reported the events' definitions: \textit{dropped-packages} and \textit{batch-completed}.

When the scale measures the weight of the packages in the batch, some packages may be too much heavy or too light.
In this case, the scale drops the packages and raises the \textit{dropped-packages} event.
This event is captured by the service that manages all the reports to create a report about the dropped packages.
The event carries information about the batch that has been checked and the number of packages that have been dropped.
This type of event is captured by the stocking service that determines if the batch is still usable or if it has to be discarded.

When the scale has finished measuring the weight of the packages in the batch, it raises the \textit{batch-completed} event.
The event carries information about the batch that has been checked and the mean and the dropped packages in the batch.
The logic about approving o rejecting a batch is delegated to the stocking service.
