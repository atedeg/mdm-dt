\chapter{Digital Twins description}

\section{Digital twins for machines}
\subsection{Milk Tank}
\subsubsection{Thing Model}
Properties: Manufacturer, Serial number, pH, minPH, maxPH, temperature, minTemp, maxTemp, quantityOfmilk
Events: temperatureOutOfRange, pHoutOfRange
Actions: Open Valve

\section{Digital twins for anticipazione guasti}
\subsection{Packaging machine}
\subsubsection{Thing Model}
Properties: Manufacturer, Serial number, currentBatch, cutterTemperature
Events: packagingStarted, packagingMachineFailure, packageDamaged,
Action: startPackaging

% cutterTemperature viene passato all'algoritmo che in caso di packagingMachineFailure o packageDamaged poi vede che magari la temperatura era troppo alta e capisce qual era il problema

riceve le info su quale lotto deve iniziare a confezionare così da impostare automaticamente i parametri della macchina.
tramite algoritmo di machine learning, la macchina predice il guasto e lo comunica tramite alert.
I parametri che valuta sono: numero di chiusure al minuto per tipo di prodotto, numero di confezioni danneggiate, temperatura resistenza saldatore (cutter).

\section{Digital twins for quality assurance}
DT a livello di processo. Quello per la QA(metal detector e scale) viene creato e distrutto all'interno di un ciclo di valutazione di un lotto.
Si collega ai DT del metal detector e della bilancia per effettuare la QA.
\subsection{Metal detector}
\subsubsection{Thing Model}
Properties: Manufacturer, Serial number, currentBatch
Events: numberOfDroppedPackages

Rifiuta il batch se almeno un prodotto viene rilevato come non conforme, altrimenti invia il batch al successivo step.
I parametri sono:
numero di prodotti rilevati come non conformi appartenenti ad un determinato lotto

\subsection{Scale}
% facciamo un capitolo "Motivazioni" dove descriviamo tutti questi scenari che giustificano l'inserimento di un DT. 
% Esempio nel caso della bilancia, scenario:
% con il tempo si osserva che un determinato prodotto inizia ad avere mediamente un peso maggiore o inferiore rispetto al peso di riferimento. Questo potrebbe essere sintomo di una problematica nel confezionamento (es. formine deformate).

\subsubsection{Thing Model}
Properties: Manufacturer, Serial number, currentBatch, meanWeight, stdDevWeight
Events: numberOfDroppedPackages, batchCompleted

Approva in automatico il batch se tutti i prodotti vanno bene, altrimenti rifiuta tutto il batch obbligando l'operatore ad intervenire manualmente.
I parametri sono:
per ogni lotto il numero di confezioni pesate, il numero di confezioni che superano il peso minimo e il numero di confezioni che superano il peso massimo, il peso medio per lotto, la deviazione standard per lotto
